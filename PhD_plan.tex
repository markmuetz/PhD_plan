\documentclass[11pt,a4paper]{article}

\usepackage{siunitx}
\usepackage{mhchem}

\usepackage[pdftex]{color,graphicx}
\pagestyle{plain}
\usepackage{geometry}
\newgeometry{margin=1.5cm}

\newcommand{\ts}{\textsuperscript}
\newcommand{\ic}{\texttt}

%\usepackage[backend=biber,style=authoryear,sorting=nyt,dashed=false]{biblatex}
%\renewcommand*{\nameyeardelim}{\addcomma\space}
%\addbibresource{references/references.bib} % note the .bib is required

% REM: null case - no spurious organization.

% terminology:
% iUM: Idealized UM
% MONC: Met Office NERC Cloud Model
% PC: Plant-Craig scheme
% GR: Gregory-Rowntree scheme
% KF: Kain-Fritch scheme
% GZ: Grey-zone
% LS: Large-scale

\begin{document}

\begin{center}
    \Large{\textbf{PhD Plan}}\\[0.15cm]
    \large{Mark Muetzelfeldt 3/5/17 [6\ts{th} Draft]}\\ 
    \rule{\textwidth}{0.2mm}
\end{center}

\section*{Overview}
% iUM overview/science/questions

The overall goal of the project is to develop scale-awareness in convective cloud parametrization schemes. More specifically, I will be looking at deep cumulus convection, using idealized RCE models over a bi-periodic domain [before moving on to investigating it in global models]. I will look at how deep cumulus parametrization schemes can be made scale-aware when going from coarse resolutions to finer resolutions in the convective grey zone. To meet this goal, I will try to answer the following questions:

\begin{enumerate}
% What is causing the organization?
    \item What effect does convective organization have on the spatial and temporal variability of idealized RCE models?
    \item How can characterization of this convective organization be used to inform the design of a scale-aware parametrization scheme?
    \item What are the main mechanisms that cause this convective organization?
    \item How can information about fluxes in a high-resolution idealized model be incorporated into a lower-resolution model's parametrization scheme?
    \item  {[How does this scheme behave and perform when it is used in global simulations?]}
\end{enumerate}

The following phases form a road-map of how I intend to go about answering these questions. Phase 0 will put the numerical model through its paces, allowing me to have confidence in its use for the subsequent phases. Phase 1 will address Q1, using a high-resolution model to perform experiments that shed light on the convective organization in idealized models. Phase 2 will address Q2 and Q3, using lower-resolution simulations to test the parametrization scheme in an idealized environment. Phases 1 and 2 combined will allow me to investigate Q4 by seeing how to use information from the high-resolution simulations to improve the lower-resolution parametrization schemes. [Phase 3 will address Q5, using the UM to run global models.]

Each phase comes with deliverables: reports, experimental results or code that are necessary for the completion of that phase. The reports will be based on some aspect of the analysis carried out as part of the phase, and will provide useful material in the form of figures and text when writing up my thesis. The experimental results will be runs of numerical models. The code will be the analysis code that I am using to run all the post run analysis - such as calculating momentum fluxes from the $u$ wind increments. I will keep all of this in a code project called archer\_analysis. Additionally, I will make a tool called omnium, which will provide tools for e.g. converting from .pp1 to to .nc files and viewing the data.

\section*{Phase 0 - iUM checks and balances}
% Uses a prescribed cooling model
% domain mean precip/surf fluxes
% vertical profiles of moisture/energy/heating
% investigate moisture non-conservation
% investigate how turning on moisture conservation works - diag which cells are affected
This phase will test the numerical model under study: the iUM. The modelling will be done with the iUM in an idealized RCE setup, with prescribed cooling. These tests will include checking the moisture and energy balance of the model to confirm that their conservation requirements are met or explain where they are violated. This will be done by looking at e.g. domain mean precipitation/surface fluxes. I will also investigate the vertical profiles of the moisture and temperature variables and increments/tendencies. A key part will be looking into the Moisture Conservation (MC) scheme, to see how it affects the model and where it is most active in ensuring conservation of variables. I will also investigate how the model behaves with MC turned off to get a handle on where the non-conservation of moisture (and possibly energy) is most problematic.

\subsection*{Time frame}

This phase was finished by the start of February 2017 on hand over of conservation property checking duties to Todd Jones. (Originally meant to be January 2017.)

\subsection*{Deliverables}
\begin{enumerate}
    \item Experiments: energy/moisture balance in high-resolution simulations. MC on/off with a full, useful and packaged set of diagnostics that allow for easy switching on/off of various options.
    \item Report: detailed model setup: base domain/timestep etc. and which schemes are in use.
    \item [Report: energy/moisture conservation in the iUM: how it is affected by MC and where it is worst.]
\end{enumerate}

\section*{Phase 1 - Convective organization in high-resolution RCE models}
% iUM (+MONC?)
% cell/cloud tracking (integrate or dev)
% cloud stats: lifetimes, locations, spacing
% varying organization: changing windshear
This phase will use the high-resolution model tested in the previous phase and begin to answer the questions posed in the overview. As well as the analysis from before, I will investigate the statistics of the clouds produced in the model, which will require the development or integration of existing cloud classification/tracking tools. These statistics will include the lifetime, location and spacing of the clouds in the models, as well as detailed analysis of the cloud such as their vertical profiles at different points in their life-cycle, and aggregate statistics of e.g. total precipitation over their lives.

% investigation of the effects of upscaling to various res
% diagnose *resolved* fluxes of heat, moisture ($\rho q w$ etc.)
The analysis will include looking at the effects of upscaling the model to various resolutions, and the diagnosis of fluxes of heat and moisture ($\rho q w$ etc.). The proposed mechanism for changing the degree of convective organization is to vary the windshear across the domain.


\subsection*{Time frame}

This phase should be finished by July 2017. (Originally May 2017)

\subsection*{Deliverables}
% lit review of high-res modelling
% detailed reports into:
% * investigation of upscaling
% * diagnosis of how cloud stats are affected by organization
% * unresolved (at upscaled res) fluxes
% * fully interactive vs prescribed rad. RCE
% * [analytical/simplified cloud model]
% omnium beta + cloud tracking code
% Phase 2 detailed plan
\begin{enumerate}
    \item Experiments: high-resolution experiments.
    \item Report: literature review of high-resolution modelling. (Finished by MC4, July 2017)
    \item Report: how cloud statistics are affected by degree of organization.
    \item Report: analysis of fluxes of energy/moisture. (What will this contain?)
    \item Code: upscaling and variability analysis.
\end{enumerate}

\section*{Phase 2 - Convective organization in grey-zone RCE models}
% run at variety of res, from fully p'trized to ~explicit conv.
% run at LS with >1 scheme: GR, KF, PC
% run in GZ with PC
% mod to PC based on diag of organization
% comparison with equiv high-res runs
% investigate scale-awareness of schemes
(Here things get more speculative and subject to change) Run at a variety of resolutions, from fully parametrized down to explicitly resolved deep cumulus convection. Run with a number of schemes: GR, KF, PC. Run in GZ with PC. Modification the PC based on diagnosis of organization in Phase 1. Comparison with equivalent high-resolution runs. Investigate how the schemes perform in terms of scale-awareness.

\subsection*{Time frame}

This phase should be finished by December 2017. (Originally October 2017.)

\subsection*{Deliverables}
% lit review of parametrized modelling/scale-awareness and parametrization
% investigation of the dependence of parametrized models on resolution
% comparison of different schemes above, and how well they compare to high-res
% Phase 3 plan
\begin{enumerate}
    \item Experiments: series of experiments run at various resolutions, with a variety of schemes.
    \item Report: literature review of parametrized modelling, scale-awareness and stochastic parametrization schemes.
    \item Report: investigation of the dependence of parametrized models on resolution.
    \item Report: comparison of different schemes above, and how well they compare to high-resolution runs.
    \item Code: omnium version 1.0.
\end{enumerate}

%\section*{Phase 3 - Testing behaviour and performance of parametrization scheme in a global model}

%[TBD]

\end{document}