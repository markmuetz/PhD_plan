\documentclass[11pt,a4paper]{article}

\usepackage{siunitx}
\usepackage{mhchem}
\usepackage{pgfgantt}
\usepackage[normalem]{ulem}

\usepackage[pdftex]{color,graphicx}
\pagestyle{plain}
\usepackage{geometry}
\newgeometry{margin=1.5cm}

\newcommand{\ts}{\textsuperscript}
\newcommand{\ic}{\texttt}

%\usepackage[backend=biber,style=authoryear,sorting=nyt,dashed=false]{biblatex}
%\renewcommand*{\nameyeardelim}{\addcomma\space}
%\addbibresource{references/references.bib} % note the .bib is required

% REM: null case - no spurious organization.

% terminology:
% iUM: Idealized UM
% MONC: Met Office NERC Cloud Model
% PC: Plant-Craig scheme
% GR: Gregory-Rowntree scheme
% KF: Kain-Fritch scheme
% GZ: Grey-zone
% LS: Large-scale

\begin{document}

\begin{center}
    \Large{\textbf{PhD Plan}}\\[0.15cm]
    \large{Mark Muetzelfeldt 7/12/2017 [8\ts{th} Draft]}\\ 
    \rule{\textwidth}{0.2mm}
\end{center}

\section*{Overview}
% iUM overview/science/questions

%The overall goal of the project is to develop scale-awareness in convective cloud parametrization schemes. More specifically, I will be looking at deep cumulus convection, using idealized RCE models over a bi-periodic domain [before moving on to investigating it in global models]. I will look at how deep cumulus parametrization schemes can be made scale-aware when going from coarse resolutions to finer resolutions in the convective grey-zone. To meet this goal, I will try to answer the following questions:
The goal of this project is to represent cloud field organization in a convective parametrization scheme, and to do this in a way that is scale-aware. I will focus on shear-induced organization, such as e.g. squall lines. I will use high-resolution idealized CRM simulations of RCE driven by different shear profiles to induce cloud field organization. I will measure various aspects of this cloud field, such as cloud lifetime, bulk entrainment rate and its spatio-temporal variation. This information will be used to make modifications to a convective parametrization scheme, based on a diagnosis of shear in the scheme. The results of using the modified scheme will be analysed, both in an idealized settings and in a global CRM. Scale-awareness will be provided by making the changes using the Plant-Craig scheme.

\begin{enumerate}
% What is causing the organization?
    \item How can shear profiles be effectively characterized in GCMs?
    \item What effect does shear-induced convective organization have on idealized RCE models?
    \item What are the main mechanisms that cause this convective organization?
    \item How can characterization of this convective organization be used to inform the design of a shear-aware parametrization scheme?
    \item  How does this scheme behave and perform when it is used in idealized simulations?
    \item  How does this scheme behave and perform when it is used in global simulations?
\end{enumerate}

\subsection*{Timetable}

\begin{figure}[hbp!]
    \begin{ganttchart}[vgrid, hgrid, y unit chart=0.75cm, MC/.style={milestone/.append style={shape=circle}}]{1}{20}  % <---
    \gantttitle{2017}{2}
    \gantttitle{2018}{12}
    \gantttitle{2019}{6} \\
    \gantttitle{N}{1}
    \gantttitle{D}{1}
    \gantttitle{J}{1}
    \gantttitle{F}{1}
    \gantttitle{M}{1}
    \gantttitle{A}{1}
    \gantttitle{M}{1}
    \gantttitle{J}{1}
    \gantttitle{J}{1}
    \gantttitle{A}{1}
    \gantttitle{S}{1}
    \gantttitle{O}{1}
    \gantttitle{N}{1}
    \gantttitle{D}{1}
    \gantttitle{J}{1}
    \gantttitle{F}{1}
    \gantttitle{M}{1}
    \gantttitle{A}{1}
    \gantttitle{M}{1}
    \gantttitle{J}{1} \\
    \ganttmilestone[MC, milestone left shift=0.2,milestone right shift=-0.2]{Monitoring committees}{2} 
    \ganttmilestone[MC, milestone left shift=0.2,milestone right shift=-0.2]{}{8} 
    \ganttmilestone[MC, milestone left shift=0.2,milestone right shift=-0.2]{}{14}  \\
    \ganttbar{Shear climatology (P1)}{2}{3} \\
    \ganttbar{High-resolution modelling (P2)}{3}{5} 
    \ganttmilestone{}{5} \\
    \ganttbar{Paper writing (P2)}{4}{8} 
    \ganttmilestone{}{8} 
    \ganttbar{}{12}{15} \\
    \ganttbar{Idealized parametrized (P3)}{9}{11} \\
    \ganttbar{Global parametrized (P4)}{11}{13} 
    \ganttmilestone{}{13} \\
    \ganttbar{Thesis writing (P5)}{6}{17} 
    \ganttmilestone{}{17} 
\end{ganttchart}
\caption{Gantt chart showing remaining timetable and tasks. Milestones shown as diamonds. The high-resolution modelling finished by EGU (start of April 2018), and the remaining milestones are near to MCs, allowing for progress checks at MC VI and MC VII. Thesis writing will ramp up starting in Q2 next year (or possibly slightly before). The finish date is 1\ts{st} April, 2019. Phases are shown in brackets.}
\label{gantt}

\end{figure}

The following phases form a road-map of how I intend to go about answering these questions. Phase 0 will put the numerical model through its paces, allowing me to have confidence in its use for the subsequent phases. Phase 1 will address Q1, by running parametrized climate simulations to look into the range of shear profiles generated by the model. Phase 2 will address Q2 and Q3, using a CRM/high-resolution model to perform experiments that shed light on the convective organization in idealized models. Phase 3 will address Q4 and Q5, using lower-resolution simulations to test the parametrization scheme in an idealized environment. Phase 4 will address Q6, using the UM to run global models. In phase 5 I will tie together what I have discovered into a coherent thesis on how shear-induced organization can be represented in a parametrized model.

The milestones are intended to lie close to important events, e.g. EGU in April 2018, MC6 and MC7. This will allow for interesting results for EGU and progress checks along the way.

Each phase comes with deliverables: reports, experimental results or code that are necessary for the completion of that phase. The reports will be based on some aspect of the analysis carried out as part of the phase, and will provide useful material in the form of figures and text when writing up my thesis. The experimental results will be runs of numerical models. The code will be the analysis code that I am using to run all the post run analysis - such as calculating momentum fluxes from the $u$ wind increments. I will keep all of this in a code project called suitably revisioned code repositories. Additionally, I will make a tool called omnium, which will provide tools for e.g. converting from .pp1 to to .nc files and viewing the data.

\section*{Phase 0 - iUM checks and balances}
% Uses a prescribed cooling model
% domain mean precip/surf fluxes
% vertical profiles of moisture/energy/heating
% investigate moisture non-conservation
% investigate how turning on moisture conservation works - diag which cells are affected
This phase will test the numerical model under study: the iUM. The modelling will be done with the iUM in an idealized RCE setup, with prescribed cooling. These tests will include checking the moisture and energy balance of the model to confirm that their conservation requirements are met or explain where they are violated. This will be done by looking at e.g. domain mean precipitation/surface fluxes. I will also investigate the vertical profiles of the moisture and temperature variables and increments/tendencies. A key part will be looking into the Moisture Conservation (MC) scheme, to see how it affects the model and where it is most active in ensuring conservation of variables. I will also investigate how the model behaves with MC turned off to get a handle on where the non-conservation of moisture (and possibly energy) is most problematic.

\subsection*{Time frame}

This phase was finished by the start of February 2017 on hand over of conservation property checking duties to Todd Jones. (Originally meant to be January 2017.)

\subsection*{Deliverables}
\begin{enumerate}
    \item Experiments: energy/moisture balance in high-resolution simulations. MC on/off with a full, useful and packaged set of diagnostics that allow for easy switching on/off of various options.
\end{enumerate}

\section*{Phase 1 - Shear climatology}

This phase will form an important link between the shear profiles in the GCM and the high resolution modelling, by shedding some light on when certain shear profiles occur, and how I might go about characterizing them. It will also look into how I might diagnose the shear in a grid-column in a way that means that the convective parametrization can respond to it, making the scheme `shear aware'.

\subsection*{Time frame}

This phase will be finished by February 2018.

\subsection*{Deliverables}
\begin{enumerate}
    \item Means of characterizing shear in GCMs
    \item Climatology of shear in GCMs
    \item {[Set of RWPs]}
\end{enumerate}

\section*{Phase 2 - High-resolution modelling}
% iUM (+MONC?)
% cell/cloud tracking (integrate or dev)
% cloud stats: lifetimes, locations, spacing
% varying organization: changing windshear
This phase will use the CRM tested in phase 0 and begin to answer the questions posed in the overview. As well as the analysis from before, I will investigate the statistics of the clouds produced in the model, which will require the development or integration of existing cloud classification/tracking tools. These statistics will include the lifetime, location and spacing of the clouds in the models, as well as detailed analysis of the cloud such as their vertical profiles at different points in their life-cycle, and aggregate statistics of e.g. total precipitation over their lives.

The organization in these simulations will be diagnosed, using an area-weighted normalized histogram of cloud to cloud spacing. It will also look at the mechanisms behind the organization, as well as some parameters such as the bulk entrainment, and these parameters' dependence of the diagnosed organization.

% investigation of the effects of upscaling to various res
% diagnose *resolved* fluxes of heat, moisture ($\rho q w$ etc.)
The analysis will include looking at the effects of upscaling the model to various resolutions, and the diagnosis of mass fluxes. The proposed mechanism for changing the degree of convective organization is to vary the windshear across the domain.

The analysis will then use high-resolution models (\SI{250}{m}), to improve the representation of the convective cells. The analysis carried out will be similar to the above. 

\subsection*{Time frame}

The modelling, analysis and results from this phase will be complete by April 2018 (Milestone 1). This phase is scheduled to finish by July 2018 (Milestone 2). [Draft 8: This phase is the merger of phases 1 and 2 from previous drafts. It has been moved to after the new phase 1 due to being dependent on what I find in that phase.]

\sout{Draft 7: This phase should be finished by July 2017. (Originally May 2017)}

\sout{Draft 7: This phase should be finished by September 2017.}

\subsection*{Deliverables}
% lit review of high-res modelling
% detailed reports into:
% * investigation of upscaling
% * diagnosis of how cloud stats are affected by organization
% * unresolved (at upscaled res) fluxes
% * fully interactive vs prescribed rad. RCE
% * [analytical/simplified cloud model]
% omnium beta + cloud tracking code
% Phase 2 detailed plan

Key deliverable: Paper on shear induced cloud field organization in high-resolution simulations.

\begin{enumerate}
    \item Experiments: CRM experiments.
    \item Experiments: High-resolution experiments.
    \item Report: Draft paper - review of CRM, how cloud statistics are affected by degree of organization. (Finished by MC4, July 2017)
    \item Code: upscaling and variability analysis.
    \item Draft paper: review of high-resolution modelling. Comparisons with CRM simulations.
    \item Code: upscaling and variability analysis. (High-res capable)
\end{enumerate}

\section*{Phase 3 - Convective organization in parametrized idealized RCE simulations}
Run at a variety of resolutions in the soft grey-zone. Run with a number of schemes: GR, KF, PC. Run in GZ with PC. Modification of GR/PC based on analysis of high-resolution simulations in Phase 2. Specifically, look into how lifetime statistics and bulk entrainment rates change with organization/wind-shear in simulations. Comparison with equivalent CRM/high-resolution runs. Investigate how the schemes perform in terms of scale-awareness.

\subsection*{Time frame}

This phase should be finished by October 2018.

\sout{Draft 7: This phase should be finished by Q1 2018.}

\subsection*{Deliverables}
% lit review of parametrized modelling/scale-awareness and parametrization
% investigation of the dependence of parametrized models on resolution
% comparison of different schemes above, and how well they compare to high-res
% Phase 3 plan
\begin{enumerate}
    \item Experiments: series of experiments run at various resolutions (\SI{10}{km}-\SI{50}{km}), with a variety of schemes.
    \item Report: literature review of parametrized modelling, scale-awareness and stochastic parametrization schemes.
    \item Report: investigation of the dependence of parametrized models on resolution.
    \item Report: comparison of different schemes above, and how well they compare to high-resolution runs.
\end{enumerate}

\section*{Phase 4 - Convective organization in parametrized global simulations}

Investigate how modified parametrization scheme performs in a GCM. Look at the climatology (expect to see degradation for a non-tuned scheme). Look at specific part of the simulation where organization will be important (speculatively MJO, AEW others?). 

\subsection*{Time frame}

This phase should be finished by December 2018.

\subsection*{Deliverables}

\begin{enumerate}
    \item MO standard climatology analysis.
    \item Comparison between different schemes and where they perform well/less well.
\end{enumerate}

\section*{Phase 5 - Thesis writing}

Write up what I found in the previous phases.

\subsection*{Time frame}

This phase should be finished by April 2019.

\subsection*{Deliverables}

Thesis.

\section*{Appendix}

\subsection*{Major revisions}
\begin{enumerate}
    \item 5/12/17, Draft 7 to 8: Add Gantt chart, reformulate overall goal and questions, update phases to add in link between GCM shear profiles and high-resolution simulations. Flesh out Phase 3/4 in more detail. Add in final Phase 5.
\end{enumerate}

\end{document}